% Anpassung des Seitenlayouts --------------------------------------------------
%   siehe Seitenstil.tex
% ------------------------------------------------------------------------------

% Anpassung an Landessprache ---------------------------------------------------
\usepackage[english]{babel}

% \TODO set tocdepth
\setcounter{tocdepth}{3} % 2
\setcounter{secnumdepth}{4}

% Math
\usepackage{amsmath}
\usepackage{amsfonts}

% bold greek letters
\usepackage{bm}

% euro symbol
\usepackage[official]{eurosym}

% Umlaute ----------------------------------------------------------------------
%   Umlaute/Sonderzeichen wie äüöß direkt im Quelltext verwenden (CodePage).
%   Erlaubt automatische Trennung von Worten mit Umlauten.
% ------------------------------------------------------------------------------
\usepackage[utf8]{inputenx}
% \usepackage[latin1]{inputenc}


% Grafiken
\usepackage{float}


% Einfache Definition der Zeilenabstände und Seitenränder etc. -----------------
\usepackage{setspace}
\usepackage{geometry}

% Literaturverzeichnis ---------------------------------------------------------
\usepackage{cite}
\bibliographystyle{ecta}
\usepackage{natbib} 
% \usepackage[backend=biber,style=authoryear,refsection=section]{biblatex}


% zum Umfließen von Bildern ----------------------------------------------------
\usepackage{floatflt}


% fortlaufendes Durchnummerieren der Fußnoten ----------------------------------
\usepackage{chngcntr}
\counterwithin{figure}{section}
\counterwithin{table}{section}




% definiert u.a. die Befehle \t'odo und \listoft'odos
\usepackage{todonotes}


\usepackage{blindtext}


% Hyperlinks
% \usepackage{hyperref}

% \usepackage{glossaries}

\usepackage{bookmark}


% Tables
\usepackage{multirow}
\usepackage{multicol}

% also for tables
\usepackage{array}