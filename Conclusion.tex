\section{Conclusion}
\label{CH:Conclusion}

In this thesis we introduced and investigated the novel approach of a plasticity based weighting of a set of experts which individually are Echo State Networks from the field of reservoir computing. This idea was build on different ideas like the Echo State Incremental Gaussian Mixture \citep{Engel2010IncrementalGaussianMixtures, Heinen2011IGMN, Heinen2011ACA,ESIGM2011}, the Gaussian Mixture Autoregressive Model \citep{Kalliovirta2015GMUnivariateSeries} and the plasticity tuning of echo state network dynamics \citep{Schrauwen2008}. Especially the latter, where the authors tune the dynamics, namely the activation values of the reservoir, of an Echo State Network towards a normal distribution $\mathcal{N}(\mu, \sigma^2)$. Alltogether, theses individual ideas lead to the approach of weighting a set of ESN experts by the plasticity, this is the likelihood, of their network activation states which makes intuitive sense because it relates to information maximizatino of the network's activation values. This way enables an updating of the weights prior to the prediction step of the experts approach which is contrast to the more traditional weighting of a set of experts based on a loss function. For the weighting based on a loss function, the true values has to be observed and the weights can be adjusted \textit{afterwards}. Under some assumptions that build on the successfull pre-tuning of network dynamics towards a targeted normal distribution, we showed the close connection of the two approaches. More specifically, the plasticity weighting can be regarded as an update of the weights based on the expected loss of an expert.

In the application to the prediction of daily realized volatilities of the IBM stock price in the years between 2001 and 2018, we presented the predictive capabilities of this new weighting approach in comparison to the loss induced weighting of experts as well as a standard single Echo State Network of comparable size or the famous Heterogeneous Autoregressive Model (HAR) model. We examined two different prediction methodologies, namely a fixed training prediction and a rolling training prediction. Both, the loss induced experts and the plastiticy weighted experts were not able to outperform both of their comparators, where the single Echo State Network was able to outperform the HAR in the short term predictions.  The fact that the weight distributions of the expert models from the two prediction methodologies didn't differ substantially but presented a very similar picture of weight distributions, led to the conclusion, that the expert methods were able to account for changes in the short and long-term dependencies in the series. Re-estimating the output connections even increased the prediction error in some cases. Focussing on the expert settings, the plasticity approach seemed to have slightly outperformed the loss induced weighting of experts. Given that this improvement was minor, further analysis, such as the model confidence set, could be conducted in this direction.

One direction of diving deeper into the plastiticy weighting of experts would be a more analysis based approach to choosing appropriate output distributions which may be task dependent or possible even would have general capabilities. Other directions may include the following: Firstly, a combination of the loss induced and the plasticity weighting of experts would be interesting way of extending the plasticity approach and to see whether the pre-prediction update of weighting coincides with the post-prediction update of weights based on a loss function. Secondly, plasticity weighted networks may be further enhanced by using a weighting scheme in the estimation of output weights that is based on the networks likelihood. This may account for heteroscedasticity in the network states and could potentially improve the estimation of output weights in Echo State Networks.

\newpage